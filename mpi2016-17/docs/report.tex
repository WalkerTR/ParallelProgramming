\documentclass{article}

\usepackage{hyperref}
\usepackage{graphicx}
\usepackage{amsmath}



\hypersetup{
    colorlinks = true,
    linkcolor = blue,
    anchorcolor = blue
}

\graphicspath{{img/}}





\title{Parallelization of the Game of Life in OpenMPI}

\author{Davide Dal Bianco \\ 2598719}

\begin{document}

\maketitle

\section{Introduction}
The Game of Life has been invented by the mathematician Conway in 1970 where, provided an initial state, the successive states can be computed without any external interaction. Each state consists of a matrix where each cell represents an organism and can be either 0 (dead cell) or 1 (live cell). The successive state can be computed by updating each cell according to the following conditions:
\begin{itemize}
    \item If a live cell has less then two adjacent live cells, it will die.
    \item If a live cell has two or three adjacent live cells, it will survive.
    \item If a live cell has more then three adjacent live cells, it will die.
    \item If a dead cell has exactly three adjacent live cells, it will became live.
\end{itemize}
This report aims to summarize the techniques and the algorithm used to parallelize the game of life, starting from the provided sequential version. A basic parallel version is first described and tested and then the program is improved to reach the maximum speedup possible. Though the provided sequential program is not the optimal one, the parallel one reflect its main structure in order to provide a fair speedup value.

\section{Parallel algorithm} \label{sec:parallelalgorithm}
The sequential algorithm compute the new state by updating each cell according to the previous rule. It follows that, for computing a single cell, only the adjacent ones are needed. For this reason, the entire matrix can be distributed among the processes according to a precise partition scheme. The rows partitioning should be the most suitable for this application, which consists in assigning groups of contiguous rows to different processes. This choice is better explained in the paragraph \nameref{sec:blockpartitioning}. \\
Assuming a matrix $N \times M$ and $P$ processes, each process has $\mathcal{O}(N / P)$ rows to compute for each step, thus $\mathcal{O}((M \times N) / P)$ cells. In order to compute a block, a process needs the rows that precede and succeed its block, which are owned by another process. For this reason each step introduce a communication of $\mathcal{O}(M)$. The computation comunication ratio is therefore $\mathcal{O}((M \times N) / (P \times M))$, hence the algorithm can perform well only when $P << N$. When the problem size is much greater then the number of rows, the ratio is $\mathcal{O}(N)$ and algorithm should have a nearly perfect linear speedup.


\subsection{About blocks partitioning} \label{sec:blockpartitioning}
The parallel algorithms uses the rows partitioning scheme to split the matrix among the processes. It differs from blocks partitioning because the blocks width is fixed to the width of the matrix. It can be seen that, when the number of processes is prime, the blocks partitioning coincide with rows partitioning (or columns partitioning). More complex partitioning scheme can be used, for example merging adjacent blocks, but this might lead to load imbalance problems. \\
Suppose the blocks form a $n \times m$ matrix, hence $P = n \times m$. In both partitioning schemes the computation is the same and is the size of the entire matrix divided by the number of processes, that is $\mathcal{O}((M \times N) / P)$. The amount of communication however might be slightly different. For rows partitioning we already argued that communication is $\mathcal{O}(M)$ and, to be precise, it is exactly $2M$ for each iteration. On the other hand, in case of blocks partitioning, the amount of communication needed is $2N/n + 2M/m + 4$, that is, it highly depends on the choice of $P$. For example, if $P = 13$ than $m = 1$ or $n = 1$ and the communication required by rows partitioning is fewer. If $P = 16$, we can choose $m = 4$ and $n = 4$ and the communication required by blocks partitioning is fewer. Finally, if we have $P = 17$ we are in the first case again and the communication required by rows partitioning is fewer. \\
Finally, using blocks partition scheme, each process has to exchange messages with 8 different processes, whilst the rows partitioning scheme requires to communicate with only 2 other processes. \\
We can conclude that blocks partitioning performs better when the number of processes is big and it is not a prime number. For these reasons, the rows partitioning is the most suitable for this application.


\subsection{Rows assignment} \label{sec:rowsassignment}
The rows partitioning scheme should assign the same number of rows to each process, in order to avoid load imbalance problems. However, this is possible only when $N$ is a multiple of $P$. When such condition is not verified, we should try to obtain the minimum load imbalance possible, that is, the number of rows per process should be $\lfloor N/P \rfloor$ or $\lceil N/P \rceil$. Given $S = (N~mod~P) \times \lceil N/P \rceil$, the following function $f: Nat \to Nat$ assign each row to its process:
\[
f(x)=
\begin{cases}
x / \lceil N/P \rceil & if~x < S \\
(x - S) / \lfloor N/P \rfloor + r & if~x \geq S \\
\end{cases}
\]
This function assign $\lceil N/P \rceil$ rows to the first $N~mod~P$ processes and $\lfloor N/P \rfloor$ to the other ones. $S$ represents the switch point between the two subfunctions and correspond to the number of rows assigned to the first $N~mod~P$ processes. In other words, the function $f$ is a broken line where the slope in the first part is less then the slope in the second part. When $N~mod~P$ is zero, then $S$ is zero and the function is a straight line that assign the same number of rows to each process.

\section{Implementation}
The parallel program has been implemented starting from the sequential one and they share most of the code. The process with rank 0 is the master node and it also create the matrix and collect the final result. The sequential program has been modified in this way:
\begin{itemize}
    \item Process with rank 0 create the matrix and, according to the function defined in the paragraph \nameref{sec:rowsassignment}, send each row to the process that owns it.
    \item Each process receive the rows it owns.
    \item For each iteration, each process exchange the boundary rows with the adjacent processes and then perform the computation.
    \item At the end of all iterations, each process compute the number of local live cells and then the number of total live cells is obtained by summing the local results with \texttt{MPI\_Reduce}.
    \item Process with rank 0 prints the global result.
\end{itemize}
After creating the matrix, process 0 use asynchronous non-blocking sends to spread data among processes. The matrix does not mute, hence it is not necessary to buffer the messages. However, before freeing the memory of the matrix, it must be ensured that all messages has been received using \texttt{MPI\_Wait}. \\
Due to different startup time and different amount of work (for process 0), some processes might start the iterations before others ones. In particular, some processes could be waiting for the boundary rows while the sending processes might still be in the startup phase. Clearly this behaviour is unwanted and it can be avoided using \texttt{MPI\_Barrier} right before starting the iterations. This synchronization point allows to start executing the iterations at the same time and avoid distortions in performance measurement. \\
During each iteration processes exchange the boundary rows. A synchronous send is highly unwanted for two reasons: unnecessary synchronization points kill performance and, for this particular communication pattern, it might be possible to reach a deadlock (problem similar to dining philosophers). On the other hand, using non-blocking sends it is not possible to update the matrix until the message has been received. It follows that a process must wait until a message is delivered and this kills the performance again. Asynchonous blocking communication is the best trade off: a small overhead is introduced for copying the message, but the state can be updated even if the messages has not been received. In order to use \texttt{MPI\_Bsend}, MPI require the user to explicitly attach the buffer. The space needed to send a message with \texttt{MPI\_Bsend} can be obtained using \texttt{MPI\_Pack\_size} and summing the constant \texttt{MPI\_BSEND\_OVERHEAD}, hence multiplying by 4 a lower bound for the buffer size can be calculated (since we use asynchronous communication a process can start the next iteration while the other one has not yet received the data in the current iteration). Note that the order of the two receive is inverted with respect to the two send to work correctly in the case there is only one process. However, after implementing this, many executions of the program ended with a buffer overflow exception. The reason can be found in the MPI implementation: though no more of four sends can be pending, there is a delay between when the message is received and when the buffer is freed. It follows that the buffering must be implemented externally with \texttt{MPI\_Isend}.
MPI communication tags are not used in the application because the communication scheme is quite easy.


\section{Performance measurement}
The parallel program should have a nearly linear speedup, as we already said in the section \nameref{sec:parallelalgorithm}. In order to measure the performance gain, the running time is measured by \texttt{MPI\_Wtime}, which returns the elapsed time on the calling processor. Once measured the local running times, we can take the maximum of them as the global running time. Choosing the maximum instead of the average provide a stricter speedup value and penalize load imbalances. However, different executions leads to much different running times. In fact, the execution time of a program running in an interactive system (and even more with many interactive systems interconnected) is always delayed by a random value. In order to mitigate the effect of random delays, we will run each program three times and we will consider the average running time. \\
After the first performance measurement a superlinear speedup has been detected. This was not suppose to happen, therefore after further investigation the reason has been found in the copy of the boundary cells. The parallel program first copies the right-left borders and then sends the top-bottom rows (included ghost cells). The sequential program, however, copies the four corner cells separately and this introduce a noticeable overhead. Computer architectures are designed to provide the maximum performance when reading memory sequentially. Accessing the memory randomly, like accessing the four corners, has a big impact on performance. For this reason, in order to respect the same structure of the parallel program, \emph{the sequential program has been modified and the four corners are now copied together with top-bottom boundaries}.

The speedup has been measured for three problem sizes:
\begin{itemize}
    \item 1000 $\times$ 1000
    \item 5000 $\times$ 5000
    \item 7500 $\times$ 7500
\end{itemize}
When comparing problem sizes, we compare only the dimensions of the matrix. In fact, the number of iterations and the running time depends linearly, therefore the speedup depends only on the size of the matrix.

\begin{figure}
\centering
\includegraphics[width=0.8\textwidth]{speedup_v1.png}
\caption{Speedup for different problem sizes}
\label{fig:speedupcomparison}
\end{figure}

The chart in \autoref{fig:speedupcomparison} clearly shows that as the problem size increase, the speedup decrease. In order to understand the reasons of this slow down, the program has been modified to keep track of the execution time of all phases. Since the instructions for monitoring each phase introduce some overhead, they are inserted in the program by the preprocessor only when the word \texttt{DEBUG} is defined. The results are summed up in \autoref{tab:timeperphasecomparison}. It is clear that the performance loss is due to the waiting time for incoming messages, which goes from 3.54\% of the time for the smallest problem size up to 37.95\% of the time for the biggest problem size. Furthermore, other phases are pure computation on data or memory copy, hence their time and the problem size depends linearly. An attemp to reduce the time of the Receive phase is described in the paragraph \nameref{sec:overlappingcommcomp}.


\begin{table}
\centering
\begin{tabular}{|l|c|c|c|}
\hline
& Size 1 & Size 2 & Size 3\\
\hline
Boundaries copy & 0.58 (0.09\%) & 0.32 (0.03\%) & 0.39s (0.03\%) \\
\hline
Send & 2.52 (0.39\%) & 0.48 (0.06\%) & 0.41s (0.03\%) \\
\hline
Receive & 22.64 (3.54\%) & 194.81 (23.01\%) & 476.71s (37.95\%) \\
\hline
Update & 469.15 (73.36\%) & 501.71 (59.26\%) & 601.32s (47.88\%) \\
\hline
Matrix copy & 144.65 (22.62\%) & 149.36 (17.64\%) & 177.17s (14.11\%) \\
\hline
\end{tabular}
\caption{Time per phase comparison on 16 nodes} \label{tab:timeperphasecomparison}
\end{table}



\section{Improvements}

\subsection{Memory optimization}
We already discussed the process 0 is the master node, which first create the matrix and then distribute it among all processes. Though this approach is correct and probably the easiest one, it introduce a huge limits on the problem size, that is, the matrix must fit in the memory of the master node. One of the most important advantages of cluster computing is the possibility of solving much bigger problems, however this program solve (faster) the same problems as the sequential one. \\
We can modify the program in such a way that the matrix must fit in the sum of the memory of all processors. In order to obtain this behaviour, the master node, after creating a row, must send it immediatly to its owner and free the memory after the delivery. A synchronous send like \texttt{MPI\_Ssend} is what we need: it blocks until the recipient reads the message and when the call returns we can immediately free the memory. It follows that a node can't send a synchronous message to itself and therefore, instead of using \texttt{MPI\_Ssend}, rows for process 0 must be copied manually. Clearly master node does not have to receive the rows with \texttt{MPI\_Recv} anymore.


\subsection{Overlapping communication and computation} \label{sec:overlappingcommcomp}
In \autoref{tab:timeperphasecomparison} we showed that as the problem size increase, the processes spend more and more time waiting for the messages. In particular, the receive phase is the only one that lower the speedup and its time increase with the problem size. In order to reduce the waiting time for incoming messages, it is possible to overlap communication and computation. In fact, the vast majority of the matrix can be computed without the information collected by the receive. Therefore we can exploit the waiting time and compute the inner part of the matrix while messages are in transit. Only when the inner matrix has been computed \texttt{MPI\_Recv} is called to read incoming messages and process the boundary rows.


\subsection{Matrix transposition}
The matrix is splitted among processes using the rows partitioning scheme, that is, the communication is $\mathcal{O}(M)$. If the input size of the problem has (much) more columns then rows, a columns partitioning scheme would be slightly better sinche it requires $\mathcal{O}(N)$ communication. The easiest way to switch from rows partitioning to columns partitioning is transposing the matrix, since it does not change the result (each cell has the same neighbours). Though, this improvement has not been implemented because the code would be less readable and this application is meant for performance testing. The problem is easy avoidable by providing the input dimensions in the opposite order.

\end{document}